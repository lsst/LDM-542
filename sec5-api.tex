\section{The API Aspect}\label{api-aspect}

The Web APIs provide language-agnostic, location-agnostic authenticated access
to LSST data.  They are intended to be used by client tools and automated
processes.  They are not optimized for direct human use via a browser or
command line URL transfer tool, although they can of course be used that way.
The available endpoints and the URL structures they accept will be documented.

The Web APIs are undergoing detailed design.  Initial prototypes exist that
provide non-VO-compliant access to catalogs, images, image mosaics, and image
cutouts.  This section describes the ultimate goals for the LSP Web APIs and
documents design features where they are already known.

\subsection{Overview - VO services and custom LSST services}\label{api-overview}

The Web APIs support standard VO protocols where possible.  They will also
support extensions to the VO protocols and additional custom LSST services that
are useful for the Portal Aspect or JupyterLab Aspect.  To the extent that it
makes sense, such extensions will be proposed to the IVOA for incorporation
into the standard protocols.

\subsection{Catalog and other tabular data access}\label{catalog-and-other-tabular-data-access}

Catalogs, metadata, provenance, and other tabular data are accessed through
the \texttt{dbserv} endpoint.

\subsubsection{TAP and other VO-compliant services}\label{tap-and-other-vo-compliant-services}

SCS and TAP will be supported.

\paragraph{ADQL implementation}\label{adql-implementation}

The ADQL implementation will support basic SQL92 (without complex subqueries)
and region specification extensions.  Slight variations in the dialect may be
visible between the "conventional" RDBMS back-end and the Qserv distributed
back-end.

\paragraph{Return data formats}\label{return-data-formats}

Returned data formats include VOTable, compressed JSON, and SQLite.

\subsubsection{Non-VO catalog interfaces}\label{non-vo-catalog-interfaces}

A custom interface will provide metadata about table columns.  See section~\{table-structure}.

\subsection{Image data access}\label{image-data-access}

Images and other files are accessed through
the \texttt{imgserv} endpoint.

\subsubsection{Image finding}\label{image-finding}

Images may be discovered via SIA data discovery or TAP query on the image metadata tables.

\subsubsection{Image retrieval}\label{image-retrieval}

Images may be retrieved via the URL returned in an SIA request or via SODA.
For convenience, key/value pairs as used by the Data Butler client library may
be provided to a custom URL interface.

\subsection{Metadata access}\label{metadata-access}

Catalog and image metadata can be accessed via TAP query to provided
metadata tables.

\subsubsection{Data Releases}\label{data-releases}

Each Data Release has its own set of tables.  These are expected to differ in
schema between Data Releases.  In addition, the image and catalog metadata
available for each Data Release will vary.  Finally, identifiers within
catalogs and images (except raw image identifiers) will vary between Data
Releases.

Accordingly, the Data Release number is a key parameter that must be
presented to the Web APIs.  Collections of datasets that are not part of
a Data Release (e.g. in-process, unreleased data or intermediate data
products) are assigned labels that are used in place of the Data Release
number.

The number of Data Releases available through the LSP and therefore the Web
APIs aspect is discussed in section~\ref{support-of-previous-releases}, 

\subsubsection{Tables}\label{tables}

Available tables include those in \citeds{LDM-153}.

\subsubsection{Table Structure}\label{table-structure}

Each table will have associated metadata that records, for each column, a
description, UCD when appropriate, unit when appropriate, and any relationship
with other columns (e.g. so several columns making up a covariance matrix can
be identified as such without resorting to pattern-matching of column names).

\subsection{User Workspace access}\label{user-workspace-access}

WebDAV and/or VOSpace will be provided for user workspace access.

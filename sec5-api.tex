\section{The API Aspect}\label{api-aspect}

The Web APIs provide language-agnostic, location-agnostic authenticated access
to LSST data.  They are intended to be used by client tools and automated
processes.  They are not optimized or supported for direct human use via a
browser, although they can be used that way.  The available endpoints and the
URL structures they accept will be documented.

The Web APIs are undergoing detailed design.  Initial prototypes exist that
provide non-VO-compliant access to catalogs, images, image mosaics, and image
cutouts.  This section describes the ultimate goals for the LSP Web APIs and
documents design features where they are already known.

The Web APIs will be accessed both from within the single-sign-on authentication
and authorization environment of the Science Platform and externally.
Internal uses will arise from the Science Platform implementation itself,
principally from the Portal Aspect, as well as from individual user activities
within the Notebook Aspect environment, accessing the Web API Aspect services
using Python interfaces.

External uses will arise from a variety of sources, including pre-built client
applications such as TOPCAT, client libraries such as PyVO and Astroquery
invoked by users in their own code, and direct programming against the
HTTPS interfaces of the Web API services.

For IVOA-standard services such as TAP, the Project will periodically test them
with available validation tools such as \texttt{taplint} \citep{TaplintLink2006ASPC..351..666T}, 
as well as with common client applications and libraries.
In particular, it is the intent of the Project to verify that the PyVO library
can be used successfully with the applicable Web API Aspect services.
(At this time, PyVO has support for TAP, basic support for DataLink, and
recently added support for SIAv2.)
If there are standards-conformance issues with PyVO, the Project may be able
to contribute effort to the improvement of the library on an open-source basis.

External users will require a means to obtain and provide Science Platform
authorization credentials ``by hand'', since their access won't be mediated by
Web-based front ends as they are in the Portal Aspect and Notebook Aspect.
The Project will provide and document straightforward means for obtaining
authorization tokens, and document examples of their use with common tools
such as TOPCAT and PyVO.
This may include the provision of ``helper functions'' to assist users in
supplying credentials.

\subsection{Overview --- VO services and custom LSST services}\label{api-overview}

The Web APIs support standard VO protocols where possible.  They will also
support extensions to the VO protocols and additional custom LSST services that
are useful for the Portal Aspect or JupyterLab Aspect.  To the extent that it
makes sense, such extensions will be proposed to the IVOA for incorporation
into the standard protocols.

\subsection{Catalog and other tabular data access}\label{catalog-and-other-tabular-data-access}

Catalogs, metadata, provenance, and other tabular data are accessed through the
\texttt{dbserv} endpoint. For metadata and provenance, table schemas, when
possible, will be reused from VO or community standards, such as CAOM2
\citep{CAOM2} and others.

\subsubsection{TAP service}\label{tap-and-other-vo-compliant-services}

\texttt{dbserv} is the LSST implementation of IVOA's TAP interface.
\texttt{dbserv} will target compliance with TAP 1.1 once it is recommended
by the IVOA.

\subsubsection{ADQL implementation}\label{adql-implementation}

Internal to dbserv is an ADQL parser. This parser will target compliance with
ADQL 2.1 once it reaches recommended status.  In addition to that, SQL language
features of underlying databases will be made available, when possible.  Some
capabilities will be restricted if a database does not support them, such as
subqueries which do not perform well in a distributed database system.

\subsubsection{Return data formats}\label{return-data-formats}

As \texttt{dbserv} will be TAP compliant, it will support the VOTable output
specified by TAP 1.1, which should be version 1.3. Additional encoding
protocols for VOTable output will be supported, including JSON and SQLite
output, which were chosen for the ability to represent metadata in-line with
VOTable data. For all output formats, gzip compression through the HTTP
protocol will be supported and encouraged.

\subsection{Image data access}\label{image-data-access}

Image discovery, metadata, and data retrieval will be available through
the \texttt{imgserv} endpoint.

\subsubsection{Image finding}\label{image-finding}

Image discovery will be facilitated through an implementation of the IVOA 
SIAv2 interface \citep{2015ivoa.spec.1223D}.
Simple implementations of SIA allow direct access to files
available via HTTP by adding a link to the SIA response; \texttt{imgserv} will
use that attribute in the SIA response to link to another service.
Image metadata access will also be facilitated through TAP, as image metadata will be
available according in the database through implementations of the IVOA ObsCore
Data Model \citep{2017ivoa.spec.0509L} and the CAOM2 data model, as well as LSST-native
image metadata representations.

\subsubsection{Image retrieval}\label{image-retrieval}

Images may be retrieved via the URL returned in an SIA request, in the case
images can be directly served over HTTP, or via a SODA implementation. The LSST
SODA implementation will also offer enhancements to implement project-specific
goals without presenting additional interfaces to science users. For example,
key/value pairs as used by the Data Butler client may be provided.

\subsection{Metadata access}\label{metadata-access}

Catalog and image metadata can be accessed via TAP query to provided
metadata tables.

\subsubsection{Data Releases}\label{data-releases}

Each Data Release has its own set of tables.  These are expected to differ in
schema between Data Releases.  In addition, the image and catalog metadata
available for each Data Release will vary.  Finally, identifiers within
catalogs and images (except raw image identifiers) will vary between Data
Releases.

Accordingly, the Data Release number is a key parameter that must be
presented to the Web APIs.  Collections of datasets that are not part of
a Data Release (e.g., in-process, unreleased data or intermediate data
products) are assigned labels that are used in place of the Data Release
number.

The number of Data Releases available through the LSP and therefore the Web
APIs aspect is discussed in section~\ref{support-of-previous-releases},

\subsubsection{Tables}\label{tables}

Available tables include those in \citeds{LDM-153}.

Catalog and image metadata tables will be implemented through IVOA and
community standards, when possible, such as those laid out by IVOA's ObsCore
DM, VOResource/RegTap, and the CAOM2 data model. Even if direct usage of those
models is not feasible, a best-effort export to them will be performed.  In
such a case, the ``native'' metadata tables will be documented for direct access
by users through TAP.

\subsubsection{Table Structure}\label{table-structure}

Each table will have associated metadata that records, for each column, a
description, UCD when appropriate, unit when appropriate, and any relationship
with other columns (e.g., so several columns making up a covariance matrix can
be identified as such without resorting to pattern-matching of column names).

\subsection{User File Workspace access}\label{user-workspace-access}

An IVOA VOSpace 2.1 service, as well as a WebDAV (IETF RFC 4918) service, will be provided for access to the User File Workspace.
(FTP access will \emph{not} be provided.)
These services will be implemented on top of an underlying filesystem provisioned as part of the LSP instance's computing resources.
Each user will have a personal ``VOSpace home'' directory in this filesystem, which will also be mounted as a POSIX filesystem in the Notebook Aspect for that user, as described in \S~\ref{user-workspace-storage} above.

\documentclass[DM,lsstdraft,toc]{lsstdoc}

% lsstdoc documentation: https://lsst-texmf.lsst.io/lsstdoc.html

% Package imports go here.

% Local commands go here.

% To add a short-form title:
% \title[Short title]{Title}
\title{Science Platform Design}

% Optional subtitle
% \setDocSubtitle{A subtitle}

\author{%
Gregory Dubois-Felsmann
and
Kian-Tat Lim
}

\setDocRef{LDM-542}

\date{\today}

% Optional: name of the document's curator
% \setDocCurator{The Curator of this Document}

\setDocAbstract{%
This document describes the design of the LSST Science Platform, the primary user-facing interface of the LSST Data Management System.
}

% Change history defined here.
% Order: oldest first.
% Fields: VERSION, DATE, DESCRIPTION, OWNER NAME.
% See LPM-51 for version number policy.
\setDocChangeRecord{%
  \addtohist{1}{YYY-MM-DD}{Unreleased.}{Gregory Dubois-Felsmann, K-T Lim}
}

\begin{document}

% Create the title page.
% Table of contents is added automatically with the "toc" class option.
\maketitle

\section{Introduction}\label{introduction}

\subsection{The Portal Aspect}\label{portal-aspect-intro}

\subsection{The Notebook Aspect}\label{notebook-aspect-intro}

\subsection{The API Aspect}\label{api-aspect-intro}

\section{The Architecture of the Science Platform}\label{architecture}

\subsection{Design Overview}\label{design-overview}

\subsubsection{Functional Architecture}\label{functional-architecture}

\subsubsection{Deployment Architecture}\label{deployment-architecture}

\subsection{Data Access and Storage}\label{data-access-and-storage}

\subsubsection{Databases}\label{databases}

\paragraph{Database content overview}\label{database-content-overview}

\begin{itemize}
\item Image and Visit metadata
\item Catalogs
\item Composite data
\item Observatory metadata (including EFD)
\item Reference catalogs
\end{itemize}

\paragraph{Database technologies}\label{database-technologies}

\begin{itemize}
\item Conventional
\item Qserv
\end{itemize}

\subsubsection{Images}\label{images}

\subsubsection{LSST-specific Objects}\label{lsst-specific-objects}

("ORM-ish" behavior, composite objects, etc.)

\subsubsection{Data Access Services}\label{data-access-services}

(This is a description of the functional architecture of the services; a detailed description of the APIs offered by the services is in section 5 below)

\subsubsection{User Catalog Data Support}\label{user-catalog-data-support}

("Level 3 catalogs")

\subsubsection{User Workspace Storage}\label{user-workspace-storage}

(Generic file-oriented storage, assumed to also support "Level 3 image data", e.g., custom coadds)

\subsubsection{Data Access Permissions and Quotas}\label{data-access-permissions-and-quotas}

\subsubsection{Support of Previous Releases}\label{support-of-previous-releases}

(This could go elsewhere in the outline hierarchy)

\subsection{Computing Resources}\label{computing-resources}

\subsubsection{Basic user compute services}\label{basic-user-compute-services}

(including the services used for the JupyterLab Python kernels)

\subsubsection{Large-scale batch and parallel computing}\label{large-scale-batch-and-parallel-computing}

\subsubsection{Resource management}\label{resource-management}

\subsection{Authentication and Authorization}\label{authentication-and-authorization}

\subsection{Cybersecurity Considerations}\label{cybersecurity-considerations}

\subsection{Additional Support Services}\label{additional-support-services}

\section{The Portal Aspect}\label{portal-aspect}

\subsection{Generic Data Browsing}\label{generic-data-browsing}

\subsection{Documentation Delivery}\label{documentation-delivery}

\subsection{Semantics-aware Workflows}\label{semantics-aware-workflows}

\subsection{Use of the LSST Stack}\label{use-of-the-lsst-stack}

\subsection{Extensibility of Visualizations}\label{extensibility-of-visualizations}

\subsection{Extensibility of the UI}\label{extensibility-of-the-ui}

\section{The Notebook Aspect}\label{notebook-aspect}

\subsection{JupyterHub / JupyterLab service}\label{jupyterhub-jupyterlab-service}

\subsection{Pre-installed LSST software}\label{pre-installed-lsst-software}

\subsection{Pre-configured access to LSST data}\label{pre-configured-access-to-lsst-data}

\section{The API Aspect}\label{api-aspect}

\subsection{Overview - VO services and custom LSST services}\label{api-overview}

\subsection{Catalog and other tabular data access}\label{catalog-and-other-tabular-data-access}

\subsubsection{TAP and other VO-compliant services}\label{tap-and-other-vo-compliant-services}

\paragraph{ADQL implementation}\label{adql-implementation}

\paragraph{Return data formats}\label{return-data-formats}

\subsubsection{Non-VO catalog interfaces}\label{non-vo-catalog-interfaces}

\subsection{Image data access}\label{image-data-access}

\subsubsection{Image finding}\label{image-finding}

\subsubsection{Image retrieval}\label{image-retrieval}

\subsection{Metadata access}\label{metadata-access}

\subsubsection{Data Releases}\label{data-releases}

\subsubsection{Tables}\label{tables}

\subsubsection{Table Structure}\label{table-structure}

\subsection{User Workspace access}\label{user-workspace-access}

(VOSpace?  WebDAV?)

\section{The Interconnectedness of the Science Platform}\label{interconnectedness-of-the-science-platform}

\subsection{Sharing Data}\label{sharing-data}

\subsection{Sharing Queries}\label{sharing-queries}

\subsection{Sharing State}\label{sharing-state}

\section{Application of the Science Platform inside the Project}\label{application-of-the-science-platform-inside-the-project}

\subsection{Developer Support}\label{developer-support}

\subsection{Commissioning}\label{commissioning}

\subsection{Observatory Operations}\label{observatory-operations}


% Include all the relevant bib files.
% https://lsst-texmf.lsst.io/lsstdoc.html#bibliographies
\bibliography{lsst,lsst-dm,refs_ads,refs,books}

\end{document}
